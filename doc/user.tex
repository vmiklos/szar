\documentclass[a4paper,12pt]{article}

\sloppy
\frenchspacing

\usepackage[left=4cm,top=4cm,right=4cm,bottom=4cm,nohead]{geometry}
\usepackage[utf8]{inputenc}
\usepackage[magyar]{babel}
\usepackage{listings}
\usepackage{multicol}
\usepackage{graphicx}
\usepackage{makecell}
\usepackage{float}

\title{Szoftverarchitektúrák (VIAUM105)\\Egyszerű verziókövető rendszer\\Felhasználói dokumentáció}
\author{Vajna Miklós (AYU9RZ)\\Veres-Szentkirályi András (YZIOAW)}

\begin{document}

\maketitle
\thispagestyle{empty}
\lstset{numbers=left, numberstyle=\tiny, basicstyle=\ttfamily, breaklines=true, frame=single, tabsize=2}

\pagebreak
\onehalfspacing
\section{Bevezetés}

A felhasználói dokumentáció célja, hogy bemutassa a felhasználó számára mind a
szerver, mind a kliens alkalmazás használatát, amennyiben annak telepítése
már a telepítési útmutató szerint megtörtént.

\section{A szerver használata}

A szerver komponens nem rendelkezik grafikus felhasználói felülettel: a kezdeti
beállításokat parancssoros interfészen keresztül lehet beállítani, majd később
a már beállított értékeket újra felhasználva nem-interaktív módban is
indítható. A szerver a következő parancssori kapcsolókkal rendelkezik:

\begin{center}
\begin{table}[H]
\centering
\begin{tabular}{| l | l | l |}
\hline
Név & Funkció & Típus \\ \hline
-ORBendPoint \emph{érték} & \makecell[l]{CORBA végpont \\ megadása, például \\ 31337-es  TCP \\ port esetén\\ giop:tcp::31337} & kötelező \\
\hline
\makecell[l]{-ORBnativeCharCodeSet \\ \emph{érték}} & \makecell[l]{CORBA karakterkódolás \\ megadása, például \\ UTF-8} & kötelező \\
\hline
-yes & \makecell[l]{korábban elmentett \\ beállítások használata, \\ interaktív mód \\ tiltása} & opcionális \\
\hline
\end{tabular}
\caption{A szerver elérhető kapcsolói}
\end{table}
\end{center}

Interaktív mód esetén indításkor a szerver a következő paraméterekre kérdez rá:

\begin{itemize}
\item MySQL host neve
\item MySQL adatbázis neve
\item MySQL felhasználó neve
\item MySQL felhasználó jelszava
\end{itemize}

A MySQL felhasználó jelszavát a szerver soha nem tárolja le biztonsági okokból,
így ha nem-interaktív módban szeretnénk használni a szervert, állítsuk be a
MySQL hozzáférést az adott felhasználóra és elérési hostra jelszó nélkül
engedélyezettre.

\end{document}
