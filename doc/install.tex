\documentclass[a4paper,12pt]{article}

\sloppy
\frenchspacing

\usepackage[left=4cm,top=4cm,right=4cm,bottom=4cm,nohead]{geometry}
\usepackage[utf8]{inputenc}
\usepackage[magyar]{babel}
\usepackage[
  unicode=true,
  colorlinks=false,
  pdfborder={0 0 0 0},
]{hyperref}

\title{Szoftverarchitektúrák (VIAUM105)\\Egyszerű verziókövető rendszer\\Telepítési útmutató}
\author{Vajna Miklós (AYU9RZ)\\Veres-Szentkirályi András (YZIOAW)}

\begin{document}

\maketitle
\thispagestyle{empty}

\pagebreak
\onehalfspacing
\section{Bevezetés}

Jelen útmutató célja a házi feladat során létrehozott szoftver telepítésének
bemutatása. A telepítési útmutatót két operációs rendszerhez készítettük el:

\begin{itemize}
\item Debian GNU/Linux (szerver és kliens)
\item Microsoft Windows NT (csak kliens)
\end{itemize}

\section{Debian GNU/Linux}

\subsection{Rendszerspecifikus telepítési folyamat}

Debian GNU/Linux rendszereken a csomagok telepítésének megszokott módja a
\verb|deb| csomagok használata. A gyakoribb eset az, hogy a felhasználó
csomagkezelőjének konfigurációja tartalmazza annak a tárolónak az elérhetőségét,
ahonnan a felhasználó a program nevének ismeretében egy kattintással telepítheti
az alkalmazást. Ritkábban a csomagfájlt tölti le a felhasználó közvetlenül,
majd azt szintén egy kattintással telepítheti.

A Debian csomagfájlban megadható, hogy az alkalmazás milyen külső könyvtárakat
igényel a futtatáshoz, a hivatalos Debian tárolók pedig tartalmazzák mindkét
függőség (omniORB és QT) legfrissebb verzióját, így a csomagfájl mérete minimális,
mindössze a metainformációkat tároló ún.\ control fájlban kell megadni az igényelt
csomagok nevét (és opcionálisan a verziószámra vonatkozó feltételeket).

\subsection{Kliens telepítésének módja}

\subsubsection{Tárolóból}

\begin{itemize}
\item Parancssorból: \verb|apt-get install sql-version-control-gui|
\item Grafikus felület (Synaptic) használatával: sql-version-control-gui kiválasztása
a listából, majd a ``Telepítés'' gombra kattintás
\end{itemize}

\subsubsection{Csomagfájlból}

\begin{itemize}
\item Parancssorból: \verb|dpkg -i sql-version-control-gui-*.deb|
\item Grafikus felület használatával: fájlböngészőben deb fájlon dupla kattintás
\end{itemize}

\subsection{Szerver telepítésének módja}

\subsubsection{Tárolóból}

\begin{itemize}
\item Parancssorból: \verb|apt-get install sql-version-control-server|
\item Grafikus felület (Synaptic) használatával: sql-version-control-server kiválasztása
a listából, majd a ``Telepítés'' gombra kattintás
\end{itemize}

\subsubsection{Csomagfájlból}

\begin{itemize}
\item Parancssorból: \verb|dpkg -i sql-version-control-server-*.deb|
\item Grafikus felület használatával: fájlböngészőben deb fájlon dupla kattintás
\end{itemize}

\subsubsection{Adatbázis séma telepítése}

A szervernek szüksége van egy MySQL adatbázisra, amelyen a következő feltételek teljesülnek:

\begin{itemize}
\item Támogatott az InnoDB tárolómotor (tranzakciók és külső kulcsok miatt).
\item Létezik az alkalmazás által igényelt séma.
\item A sémába tartozó táblákra írási és olvasási joga van az alkalmazás számára
biztosított felhasználónak
\end{itemize}

Első feltétel könnyedén teljesíthető egy pár évnél nem régebbi MySQL verzió
telepítésével, másodikhoz elég lefuttatni a szerverrel együtt terjesztett
\verb|VersionControl.sql| szkriptet. Végül a harmadik teljesítéséhez
leginkább egy, az alkalmazás számára dedikált adatbázis létrehozása,
melyre egy felhasználó teljes írási/olvasási jogosultsággal rendelkezik.

\clearpage
\section{Microsoft Windows NT}

\subsection{Rendszerspecifikus telepítési folyamat}

Windows rendszereken alkalmazások telepítésének két megszokott módja a Microsoft Installer
csomagok (\verb|msi| fájlok) ill.\ végrehajtható telepítő program használata. Előbbinek
előnye, hogy csoportházirend használatával központilag is teríthető, hátránya azonban,
hogy a létrehozására alkalmas szabad szoftverek nehezen kezelhetők, egyéb szoftverek
használatához pedig megkötéseket tartalmazó végfelhasználói szerződést kell aláírni.

Végrehajtható telepítő azonban könnyedén készíthető például a Nullsoft (Winamp fejlesztő)
által fejlesztett NSIS használatával. Ennek előnye, hogy magas tömörítési rátát alkalmazó
algoritmust (LZMA) is tartalmaz, a telepítőt pedig egy egyszerű szkriptfájlból generálja,
a Windowsos telepítő létrehozásához pedig nincs szükség Windows futtatására. Az általunk
írt script UAC-t tartalmazó Windows verziók esetén rendszergazdai jogosultságot kér, majd
felmásolja a klienst tartalmazó végrehajtható programot, a szükséges könyvtárakat és
az alkalmazás ikonját a felhasználó által választott kimeneti könyvtárba, majd létrehoz
egy parancsikont a Start menüben.

\subsection{Kliens telepítésének módja}

\begin{itemize}
\item sql-version-control-setup.exe elindítása
\item UAC esetén rendszergazdai jogosultság megadása
\item a megjelenő üdvözlőképernyőn a ``Next'' gomb megnyomása
\item kimeneti könyvtár kiválasztása, majd a ``Next'' gomb megnyomása
\item telepítés befejezése után ``Finish'' gomb megnyomása
\end{itemize}

Amennyiben a program induláskor konfigurációs hibát jelez, szükséges a kb.\ 2 MB méretű
Microsoft Visual C++ 2008 runtime csomag telepítése, ez ingyenesen letölthető a Microsoft
letöltőközpontjából\footnote{
\url{http://www.microsoft.com/downloads/en/details.aspx?FamilyID=9b2da534-3e03-4391-8a4d-074b9f2bc1bf}}.

\end{document}
