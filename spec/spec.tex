\documentclass[a4paper,12pt]{article}

\sloppy
\frenchspacing

\usepackage[left=4cm,top=4cm,right=4cm,bottom=4cm,nohead,nofoot]{geometry}
\usepackage[utf8]{inputenc}
\usepackage[magyar]{babel}
\usepackage{listings}
\usepackage{multicol}

\title{Szoftverarchitektúrák (VIAUM105)\\Egyszerű verziókövető rendszer\\Házi feladat dokumentáció}
\author{Vajna Miklós (AYU9RZ)\\Veres-Szentkirályi András (YZIOAW)}

\begin{document}

\maketitle
\thispagestyle{empty}
\lstset{numbers=left, numberstyle=\tiny, basicstyle=\ttfamily, breaklines=true, frame=single, tabsize=2}

\pagebreak
\onehalfspacing
\section{Feladat leírás}

Tegyük fel, hogy többen akarnak egy nagy adatbázist közösen megtervezni. Mivel
a modellező program nem támogatja a verzió követést, de SQL DDL scriptet vissza
tud olvasni. Ezért mindenki create script-ek formájában kiexportálja a
változtatás utáni verziót, melyet feltölt a közös tárhelybe. Ott az egyes
verziók összefésülődnek, majd mindenki letöltheti azt a verziót, mely már
minden változtatást tartalmaz. A program támogassa a kizárólagos és konkurens
hozzáférést, valamint az összefésülést. Vagyis ha két fejlesztő párhuzamosan
módosít egy create script-et, akkor mindkettő módosításai jelenjenek meg az
eredményben, vagy ha automatikusan feloldhatatlan ütközés van, akkor a program
támogassa annak feloldását.

\section{Követelmény specifikáció}
\subsection{A program informális leírása}

A program egy központosított verzikezelő rendszer funkcionalitásának
részhalmazát kívánja megvalósítani. A rendszer csak szöveges file-okat kezel,
melyeken a következő operációkat támogatja:

\begin{itemize}
\item létrehozás - ez csak akkor sikeres, ha a megadott egyedi névvel még nem létezik script
\item letöltés - már létező script tetszőleges verziójának eléréséhez
\item frissítés - ez a feltöltés azon esete, mikor már egy letöltött scriptet töltünk fel újra
\item törlés
\item zárolás - kizárólagos hozzáféréshez
\item zárolás megszüntetése
\item történet - egy script elérhető verzióit adja meg
\end{itemize}

\subsubsection{Konkurens hozzáférés támogatása}

Az ütközéseket - mikor két felhasználó is letölti ugyanazt a verziót, majd a
második felhasználó is visszatölti a scriptet - 3-way merge algoritmussal
valósítjuk meg. Ennek alapja, hogy a merge-ölendő két dokumentumnak ismert a
közös őse. Ezt úgy biztosítjuk, hogy a scripteket pozitív nemnulla integer
számokkal látjuk el, melyek a módosítások hatására növelődnek. Letöltéskor a
kliens eltárolja a letöltött script verzióját, és feltöltéskor ha a legfrissebb
verzió változott, akkor a helyi, a letöltött és a serveren lévő scriptek között
végez merge-ölést. Ha ez automatikusan nem megoldható, a kliens conflict
markereket helyez el a scriptben majd kézi feloldás után ennek eredményét tölti
a serverre.

\subsubsection{Felhasználók kezelése}

A rendszert csak jelszóval azonosított felhasználók használhatják. A
felhasználók kezeléséhez egy adminisztrációs interfész készül, mely a következő
funkcionalitással bír:

\begin{itemize}
\item felhasználó létrehozása (azonosítás email címmel)
\item felhasználó adatainak megváltoztatása (teljes név, jelszó)
\item felhasználó törlése
\end{itemize}

\section{A megvalósítás részletei}

Az alkalmazást háromrétegű architektúrával valósítjuk meg:

\begin{itemize}
\item Az adatokat MySQL adatbáziskezelőben tároljuk, adateléréshez QtSql-t használunk.
\item Az üzleti logikát (merge-ölés, stb.) egy központi helyen futó, felhasználói
felülettel nem rendelkező server alkalmazás tartalmazza, kizárólag ennek van
közvetlen hozzáférése az adatbázishoz.
\item A harmadik réteg a Qt keretrendszerrel készített grafikus felhasználói felület,
mely CORBA keretrendszer mico implementációját használja a serverrel való
kommunikációra.
\end{itemize}

A szoftvert C++ nyelven írjuk, a forráskód, tesztesetek és dokumentációt a git
verziókezelő használatával kezeljük a kollaboráció megkönnyítésé érdekében.

A rendszer legfontosabb funkcióit a cppunit keretrendszer segítségével
automatikusan tesztelhetővé tesszük.

A dokumentáció 3 formában készül el:
\begin{itemize}
\item a fejlesztői dokumentációt Doxygen megjegyzések formájában a forráskódban
helyezzük el
\item a felhasználói dokumentációt \TeX{} nyelven
\item a telepítési dokumentációt szintén
\end{itemize}

A fejlesztést Linux operációs rendszer alatt végezzük, de az összes komponens
gondosan így lett kiválasztva, hogy a szoftver hordozható maradjon, így más
UNIX vagy akár Windows rendszerre is könnyen portolható legyen.

\end{document}
